% set 0.5 inch indentation
\setlength{\parindent}{0in}
% set paragraph space = 0 space
\setlength{\parskip}{1.5mm}
% set line space 1.5
\setlength{\baselineskip}{1.6em}

\chapter{INTRODUCTION}
\section{Background}
\acrfull{vl} models have gained significant attention due to their ability to perform both zero-shot and transfer learning, achieving high performance across numerous downstream tasks through pre-training with web-scale image-text pairs \cite{s-clip, medclip, vl-review}.  
Many \acrshort{vl} models incorporated \acrfull{mlm} as a pre-training task, making it an important method to train \acrshort{vl} models \cite{albef, mplug, uniter, beit-3, lxmert}.  
Typically, a subset of word tokens is randomly masked at a percentage during training, and the model is tasked with predicting these masked tokens using information from both visual and language modalities.  
This masking approach has proven to enhance the alignment between visual and linguistic representations, boosting performance in \acrshort{vl} tasks \cite{lxmert}.  

Despite the widespread adoption of \acrshort{mlm} in \acrshort{vl} training, its effects on model performance, efficiency, and training loss remain underexplored.  
\citeA{mask_object} demonstrated that many of the randomly masked tokens are often stop-words or punctuation, which the model can easily learn without any need for masking.  
Another study by \citeA{selective_masking} demonstrated that selectively masking infrequent words from the pre-training dataset can boost model performance on out-of-domain datasets during continued training.  
Additionally, \citeA{rf-curriculum-masking} suggested that random masking causes the model to rely heavily on local text signals, and it results in inefficient and inconsistent interactions between modalities, leading to suboptimal performance.  
These findings emphasize the importance of strategic token selection in \acrshort{mlm} to enhance \acrshort{vl} model performance and efficiency.  

In this work, we aimed to address the gap in understanding how masking each \acrfull{pos} impacts \acrshort{vl} models.  
Each \acrshort{pos} contributes distinctively to sentence meaning: nouns typically denote objects, while verbs describe actions and often demand contextual comprehension.  
By selectively masking different parts of speech, we could better understand how each \acrshort{pos} category affects the alignment between visual and linguistic information.  
To further explore the effect of each \acrshort{pos}, training without the \acrshort{mlm} task and with different \acrshort{pos} masking ratios were compared.
The experiment is designed to answer the following questions:  
\begin{enumerate}  
    \item How does masking each \acrshort{pos} impact the performance, and training loss of \acrshort{vl} pre-training models?  
    \item How does each \acrshort{pos} masking strategy affect \acrfull{vqa} performance when analysed based on different question types?  
    \item What is the difference between training without the \acrshort{mlm} task compared to training with it, and when masking each \acrshort{pos} with a 100 percent masking ratio?  
\end{enumerate}


\section{Objective}  
The objectives for our experiment are as listed.  
\begin{enumerate}  
    \item Develop a pre-trained \acrshort{vl} model to evaluate the impact of masking each \acrshort{pos} on performance and training efficiency.  
    \item Benchmark the performance of each \acrshort{pos} masking approach using specialized datasets to gain a deeper understanding of masking effects with retrieval and question answering tasks.  
    \item Train the model without \acrshort{pos} masking and with \acrshort{pos} masking at a 100 percent masking ratio.  
\end{enumerate}  

\section{Scope}
\begin{enumerate}  
    \item The training and testing datasets are web-scale image-text pairs.  
    \item The model architecture is a cross-attention model, chosen for its ability to jointly predict answers based on information from multiple modalities.  
\end{enumerate}  
